\documentclass[10pt]{article}

\usepackage{../../../base/cover}
\usepackage{../../../base/head}
\usepackage{../../../base/commands}
\usepackage{../../subject-specific/head}
\usepackage{/D:/Workspace/git/latex-temp-jan24/homework/course/course-specific/course-config}

\newcommand{\blattnr}{0}

\begin{document}
    \makecover

    \begin{aufgabe}[0]
        \begin{figure}[h]
            \begin{center}
                \begin{tikzpicture}[nodes= {draw, circle}, local bounding box= bb, baseline= (bb.center), scale= 0.9, align= center, node distance= 1cm]
                    \node (S) at (0,0) {$S$};
                    \node (B) [right= of S] {$B$};
                    \node (A) [above= of B] {$A$};
                    \node (C) [below= of B] {$C$};
                    \node (D) [right= of A] {$D$};
                    \node (E) [below= of D] {$E$};
                    \node (F) [below= of E] {$F$};
                    \node (G) [right= of E] {$G$};

                    \draw[-{Stealth[length=2mm]}] (S) -> (A);
                    \draw[-{Stealth[length=2mm]}] (S) -> (B);
                    \draw[-{Stealth[length=2mm]}] (S) -> (C);

                    \draw[-{Stealth[length=2mm]}] (B) -> (A);
                    \draw[-{Stealth[length=2mm]}] (C) -> (B);

                    \draw[-{Stealth[length=2mm]}] (A) -> (D);
                    \draw[-{Stealth[length=2mm]}] (B) -> (E);
                    \draw[-{Stealth[length=2mm]}] (C) -> (F);

                    \draw[-{Stealth[length=2mm]}] (A) -> (E);
                    \draw[-{Stealth[length=2mm]}] (D) -> (G);

                    \draw[-{Stealth[length=2mm]}] (D) -> (B);
                    \draw[-{Stealth[length=2mm]}] (E) -> (C);

                    \draw[-{Stealth[length=2mm]}] (F) -> (G);

                    \draw[-{Stealth[length=2mm]}] (G) -> (E);
                \end{tikzpicture}
            \end{center}
            \caption{Graph $G$}
            \label{fig:graph-g1}
        \end{figure}

        Betrachten Sie obigen Graphen $G$ aus~\autoref{fig:graph-g1}.
        Wenden Sie jeweils das verlangte Verfahren an bzw. beantworten Sie die Frage \textbf{oder} begründen Sie, warum dies nicht geht.
    \end{aufgabe}
\end{document}